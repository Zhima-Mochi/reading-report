% =====================================================
%  Architecture & DDD — Full Slide Deck
% =====================================================
\documentclass[UTF8]{beamer}
\usetheme{Madrid}

% ----------- 套件 ----------
\usepackage{fontspec}
\usepackage{xcolor}
\usepackage{listings}
\usepackage{graphicx}
\usepackage{hyperref}
\hypersetup{colorlinks,linkcolor=black,urlcolor=blue}

\PassOptionsToPackage{quiet}{xeCJK}
\usepackage{ctex}
\setCJKmainfont{Noto Sans CJK TC}
\setCJKsansfont{Noto Sans CJK TC}
\setCJKmonofont{Noto Sans Mono CJK TC}

% ----------- 版面 ----------

\AtBeginSection[]{
  \begin{frame}
    \vfill
    \begin{center}
        \usebeamercolor{title}
        {\Huge \insertsection}
    \end{center}
    \vfill
  \end{frame}
}


% ----------- 程式碼顯示 ----------
\lstset{
  language=Java,
  basicstyle=\ttfamily\scriptsize,
  keywordstyle=\bfseries\color{blue},
  commentstyle=\itshape\color{gray},
  stringstyle=\color{purple},
  numbers=left, numberstyle=\tiny\color{gray}, numbersep=5pt,
  breaklines=true, showstringspaces=false,
  columns=flexible, tabsize=2,
  frame=single, framerule=0pt,
  backgroundcolor=\color{gray!6}
}

% ----------- 專案資訊 ----------
\title[第四章 · Architecture 與 DDD]{第四章:Architecture \\ 架構風格與 DDD 的協奏曲}
\date{\today}

% ======================================================
\begin{document}
% ------------------------------------------------------
\begin{frame}
    \titlepage
\end{frame}

% ------------------------------------------------------
\begin{frame}{目錄}
    \scriptsize
    \tableofcontents[hideallsubsections]
\end{frame}


% ================== Section: 引言 =====================
\section{開場}
\begin{frame}{開場引言}
    \begin{quote}
        \Large ''Architecture should speak of its time and place,\\
        but yearn for timelessness.'' \\
        \small - \textit{Frank Gehry}
    \end{quote}
\end{frame}

\begin{frame}{核心觀點}
    \begin{itemize}
        \item DDD 是一套以 \textbf{Bounded Context} 劃界的思維框架,而非單一實作架構。
        \item 架構 = \textbf{品質屬性 + 功能需求} 的平衡藝術。
        \item 架構應服務於 Domain 模型;勿讓技術層凌駕業務語言。
        \item 追求可演進,抵禦未知需求,才能保持系統韌性。
    \end{itemize}
\end{frame}

% ================== Section: SaaSOvation =====================
\section{SaaSOvation 演進地圖}
\begin{frame}{SaaSOvation 架構演進}
    \scriptsize
    \begin{tabular}{|c|c|c|c|c|}
        \hline
        階段       & 架構               & 驅動     & 收穫       & 風險     \\ \hline\hline
        Startup    & Monolith (Layered) & MVP 速度 & 快速迭代   & 技術債   \\ \hline
        Scale-up   & Hexagonal + CQRS   & 可測試性 & 關注點分離 & 邊界模糊 \\ \hline
        Enterprise & 多 Context + SOA   & 團隊協作 & 去耦       & 協調成本 \\ \hline
    \end{tabular}
    \vspace{0.3cm}
    \begin{block}{啟示}
        沒有一步到位的架構:\bfseries{需求 × 風險 × 演進式重構} 才是真正的長青之道。
    \end{block}
\end{frame}

% ================== Section: Layered =====================
\section{Layered Architecture}
\begin{frame}{Layered 四層模型}
    \begin{itemize}
        \item UI → Application → Domain → Infrastructure
        \item 優點:結構清晰、好上手
        \item 缺陷:若缺乏 DDD 思維,易淪為 "Controller + Service + DAO" 技術債
        \item 解方:\textbf{DIP + Ports/Adapters}、在層內劃分 Bounded Context
    \end{itemize}
\end{frame}

\begin{frame}[fragile]{Layered Java (DIP)}
    \tiny
    \begin{lstlisting}
// Domain Layer (Port)
public interface OrderRepository {
    Order findById(String id);
    void save(Order order);
}

public class Order {
    private final String id;
    private OrderStatus status = OrderStatus.CREATED;
    public Order(String id) { this.id = id; }
    public void complete() { status = OrderStatus.COMPLETED; }
    public OrderStatus status() { return status; }
}

// Application Layer
public class OrderService {
    private final OrderRepository repo; // 依賴抽象
    public OrderService(OrderRepository repo) { this.repo = repo; }
    public void completeOrder(String id) {
        Order o = repo.findById(id);
        o.complete(); repo.save(o);
    }
}
\end{lstlisting}
\end{frame}

% ================== Section: Hexagonal =====================
\section{Hexagonal Architecture}
\begin{frame}{Hexagonal 重點}
    \begin{enumerate}
        \item 所有依賴指向 Domain;外界透過 Port 呼叫
        \item Adapter 隔離協定/格式,方便測試替身
        \item 高可維護、技術棧可替換
    \end{enumerate}
\end{frame}

\begin{frame}[fragile]{Hexagonal Java 範例}
    \begin{lstlisting}
// Port
public interface PaymentPort {
    boolean pay(double amount);
}

// Domain
public class Checkout {
    private final PaymentPort port;
    public Checkout(PaymentPort port) { this.port = port; }
    public Receipt process(double amt) {
        if (port.pay(amt)) return new Receipt("OK");
        throw new PaymentFailed();
    }
}

// Adapter
public class PaypalAdapter implements PaymentPort {
    private final PaypalApi api;
    public PaypalAdapter(PaypalApi api) { this.api = api; }
    public boolean pay(double amt) { return api.execute(amt); }
}
\end{lstlisting}
\end{frame}

% ================== Section: SOA =====================
\section{SOA Architecture}
\begin{frame}{SOA 架構精要}
    \begin{itemize}
        \item 每個 Bounded Context → 獨立服務 (Service)
        \item 通訊:REST / gRPC (同步),Event Bus (非同步)
        \item 服務治理:Registry、Contract-First、Versioning、Policy Enforcement
        \item 韌性:Circuit Breaker、Retry、Bulkhead、Timeout
    \end{itemize}
\end{frame}

% ================== Section: REST =====================
\section{REST Architecture}
\begin{frame}[fragile]{Contract-First Design 範例}
    \begin{lstlisting}[language=Java]
@Path("/products")
public interface ProductService {
    @GET @Path("/{id}")
    ProductDTO get(@PathParam("id") String id);
}
\end{lstlisting}
\end{frame}

\begin{frame}{REST 架構風格}
    \begin{itemize}
        \item 資源導向,使用 HTTP 動詞 (GET/POST/PUT/DELETE)
        \item 無狀態、統一介面、可快取、分層系統
        \item HATEOAS:在回應中提供 Link 進行導覽
        \item Aggregate ↔️ Resource,API 不可破壞 Context 邊界
    \end{itemize}
\end{frame}

\begin{frame}[fragile]{REST Java 範例}
    \begin{lstlisting}[language=Java]
@RestController
@RequestMapping("/orders")
class OrderController {
    private final OrderService svc;
    OrderController(OrderService svc) { this.svc = svc; }

    @GetMapping("/{id}")
    RepresentationModel<OrderDTO> get(@PathVariable String id) {
        Order o = svc.findById(id);
        OrderDTO dto = new OrderDTO(o.status().name());
        dto.add(linkTo(methodOn(OrderController.class).get(id)).withSelfRel());
        return dto;
    }
}
\end{lstlisting}
\end{frame}

% ================== Section: CQRS =====================
\section{CQRS Pattern}
\begin{frame}{CQRS 核心概念}
    \begin{itemize}
        \item Command / Query 分離;寫入模型 ≠ 讀取模型
        \item 為寫入一致性與讀取效能分別優化
        \item 常與 Event Sourcing 搭配;需要處理最終一致性
    \end{itemize}
\end{frame}

\begin{frame}[fragile]{CQRS Java - Command Side}
    \begin{lstlisting}[language=Java]
public record CreateUserCommand(String id, String name) {}

public class UserCommandHandler {
    private final UserRepository repo;
    public UserCommandHandler(UserRepository repo) { this.repo = repo; }
    public void handle(CreateUserCommand cmd) {
        repo.save(new User(cmd.id(), cmd.name()));
    }
}
\end{lstlisting}
\end{frame}

\begin{frame}[fragile]{CQRS Java - Query Side}
    \begin{lstlisting}[language=Java]
public class UserProjection {
    @EventListener
    public void on(UserCreated e) {
        // 寫入投影 DB
    }
}

public class UserQueryService {
    private final UserReadRepo repo;
    public UserQueryService(UserReadRepo r) { this.repo = r; }
    public UserDTO fetch(String id) { return repo.find(id); }
}
\end{lstlisting}
\end{frame}

% ================== Section: Event-Driven =====================
\section{Event-Driven Styles}
\begin{frame}{事件驅動三風格}
    \begin{enumerate}
        \item Pipes \& Filters — 流式轉換
        \item Sagas — 分散式交易協調 + 補償
        \item Event Sourcing — 狀態 = 事件折疊
    \end{enumerate}
\end{frame}

\begin{frame}[fragile]{Saga 編排範例}
    \begin{lstlisting}[language=Java]
class ShippingSaga {
    @SagaEventHandler
    void on(OrderCreated e) {
        send(new ReserveInventory(e.id()));
    }
    @SagaEventHandler
    void on(InventoryReserved e) {
        send(new ArrangeShipment(e.orderId()));
    }
    @SagaEventHandler
    void on(ShipmentArranged e) {
        send(new MarkOrderShipped(e.orderId()));
        end();
    }
}
\end{lstlisting}
\end{frame}

\begin{frame}[fragile]{Event Sourcing 範例}
    \begin{lstlisting}[language=Java]
public interface DomainEvent { Instant occurredAt(); }

public record OrderCreated(String id, Instant occurredAt) implements DomainEvent {}

public class OrderAggregate {
    private String id;
    private OrderStatus status;

    public static OrderAggregate reconstitute(List<DomainEvent> history) {
        OrderAggregate agg = new OrderAggregate();
        history.forEach(agg::apply);
        return agg;
    }
    private void apply(DomainEvent e) {
        if (e instanceof OrderCreated oc) {
            this.id = oc.id();
            this.status = OrderStatus.CREATED;
        }
    }
}
\end{lstlisting}
\end{frame}

% ================== Section: Data Fabric =====================
\section{Data Fabric}
\begin{frame}{Data Fabric 架構要點}
    \begin{itemize}
        \item 統一資料平面:整合 OLTP / OLAP / Streams / Cache
        \item Smart Cache、Federated Query、Consistency Policy、Observability
        \item 與 DDD 聚合分片結合,確保資料局部性與效能
    \end{itemize}
\end{frame}

\begin{frame}[fragile]{Hazelcast 快取範例}
    \begin{lstlisting}[language=Java]
Config cfg = new Config();
HazelcastInstance hz = Hazelcast.newHazelcastInstance(cfg);
IMap<String, OrderSummary> orders = hz.getMap("orders");

OrderSummary summary = new OrderSummary("ID-123", 1023, Instant.now());
orders.set(summary.id(), summary, 30, TimeUnit.MINUTES);

Collection<OrderSummary> highValue = orders.values(
    Predicates.greaterThan("total", 1000));
\end{lstlisting}
\end{frame}

\begin{frame}[fragile]{Apache Ignite 分散運算範例}
    \begin{lstlisting}[language=Java]
Ignition.start();
Ignite ignite = Ignition.ignite();
IgniteCache<Integer, Tick> cache = ignite.getOrCreateCache("ticks");

IgniteCallable<Double> task = () -> {
    List<List<?>> rows = cache.query(
      new SqlFieldsQuery("SELECT price FROM Tick WHERE pid = ?")
        .setArgs(portfolioId)).getAll();
    List<Double> prices = rows.stream()
        .map(r -> (Double) r.get(0))
        .sorted()
        .collect(Collectors.toList());
    return prices.get((int)(prices.size() * 0.05)); // Value-at-Risk
};

Double var = ignite.compute().call(task);
\end{lstlisting}
\end{frame}

\begin{frame}{Data Fabric 風險與治理建議}
    \begin{itemize}
        \item \textbf{記憶體壓力}:熱資料量估錯 → OOM,建議 TTL + LRU Eviction
        \item \textbf{Schema 演進}:需有 Registry + 相容性驗證
        \item \textbf{Split-Brain}:多區部署需啟用 CP 模式或資料同步機制
        \item \textbf{安全}:敏感資料須加密(靜態/傳輸/使用中)與審計紀錄
    \end{itemize}
\end{frame}

% ================== 結語 =====================
\section{結語}
\begin{frame}{結語:架構即演進}
    \begin{itemize}
        \item 沒有銀彈架構,DDD 幫助你因應變化、協作清晰
        \item 每種架構風格皆服務於 Domain 模型的演進
        \item 避免技術導向的「錯配式設計」;堅持語言一致性與 Context 純度
        \item 架構設計的重點不是「選哪一種」,而是「如何隨著需求演進」
    \end{itemize}
\end{frame}

\begin{frame}[plain,noframenumbering]
    \centering
    \Huge \textbf{謝謝收看!} \\
    \vspace{1cm}
    \normalsize Slides by
\end{frame}

\end{document}
