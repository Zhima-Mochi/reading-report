% =====================================================
%  Repositories Slide Deck
% =====================================================
\documentclass{beamer}
\usetheme{Madrid}

% ----------- Packages ----------
\usepackage[T1]{fontenc}
\usepackage[utf8]{inputenc}
\usepackage{xcolor}
\usepackage{listings}
\usepackage{graphicx}
\usepackage{hyperref}
\hypersetup{colorlinks,linkcolor=black,urlcolor=blue}

\usepackage{pgfpages}
\usepackage{tikz}
\usetikzlibrary{positioning}
\setbeameroption{show notes}

% ----------- Layout ----------

\AtBeginSection[]{
  \begin{frame}
    \vfill
    \begin{center}
        \usebeamercolor{title}
        {\Huge \insertsection}
    \end{center}
    \vfill
  \end{frame}
}


% ----------- Code Listing Style ----------
\lstset{
  language=Go,
  basicstyle=\ttfamily\scriptsize,
  keywordstyle=\bfseries\color{blue},
  commentstyle=\itshape\color{gray},
  stringstyle=\color{purple},
  numbers=left, numberstyle=\tiny\color{gray}, numbersep=5pt,
  breaklines=true, showstringspaces=false,
  columns=flexible, tabsize=2,
  frame=single, framerule=0pt,
  backgroundcolor=\color{gray!6}
}
\title{Repositories in Domain-Driven Design}
\subtitle{Bridging Domain and Persistence in Go}
\author{}
\date{}

\begin{document}

\begin{frame}
  \titlepage
\end{frame}

\begin{frame}{Agenda}
  \tableofcontents
\end{frame}

\section{Why Repositories}

\begin{frame}{Bridging Domain and Data}
  \begin{itemize}
    \item Repositories mediate between domain objects and persistence, keeping the domain clean.
    \item They give the illusion of an in-memory collection while data lives in durable storage.
    \item Interfaces belong in the domain layer; infrastructure implements them behind the scenes.
    \item Promote expressive domain code such as \texttt{tenant.Activate()} without leaking database concerns.
  \end{itemize}
\end{frame}

\section{Definition and Placement}

\begin{frame}{Evans \& Vernon on Repositories}
  \begin{itemize}
    \item Evans: “A Repository mediates between the domain and data-mapping layers, giving the illusion of an in-memory collection.”
    \item Vernon reinforces the concept: expose repositories only for aggregate roots.
    \item Deliver domain objects in and out; hide records, SQL, and driver specifics.
    \item Domain layer relies on a stable contract; implementations can change via infrastructure adapters.
  \end{itemize}
\end{frame}

\begin{frame}[fragile]{Repository Interface in Go}
  \begin{lstlisting}
type TenantRepository interface {
    Add(ctx context.Context, tenant *Tenant) error
    ByID(ctx context.Context, id TenantID) (*Tenant, error)
    Remove(ctx context.Context, id TenantID) error
}
  \end{lstlisting}
  \begin{itemize}
    \item Methods reflect aggregate lifecycle operations rather than CRUD verbs alone.
    \item Keeps invariants within the aggregate while infrastructure handles serialization.
  \end{itemize}
\end{frame}

\section{Repository Styles}

\begin{frame}{Collection-Oriented Style}
  \begin{itemize}
    \item Mimics an in-memory collection; Unit of Work tracks changes automatically.
    \item Common in ecosystems with rich ORMs (Java, .NET).
    \item Works best when the tooling offers change tracking and implicit transactions.
    \item Less common in Go, but useful to understand when collaborating across stacks.
  \end{itemize}
\end{frame}

\begin{frame}[fragile]{Persistence-Oriented Style}
  \begin{lstlisting}
type TenantRepository interface {
    Save(ctx context.Context, tenant *Tenant) error
    Delete(ctx context.Context, id TenantID) error
    ByID(ctx context.Context, id TenantID) (*Tenant, error)
}
  \end{lstlisting}
  \begin{itemize}
    \item Dominant approach in Go: explicit reads and writes per call.
    \item No hidden Unit of Work—callers orchestrate state changes intentionally.
    \item Encourages clarity about I/O and transaction boundaries.
  \end{itemize}
\end{frame}

\section{Aggregate Example}

\begin{frame}[fragile]{Tenant Aggregate Governing Users}
  \begin{lstlisting}
type Tenant struct {
    ID       TenantID
    Name     string
    IsActive bool
    Users    []*User
}

func (t *Tenant) Activate() { t.IsActive = true }

func (t *Tenant) RegisterUser(email string) *User {
    user := &User{Email: email, TenantID: t.ID}
    t.Users = append(t.Users, user)
    return user
}
  \end{lstlisting}
  \begin{itemize}
    \item Tenant acts as aggregate root, enforcing invariants for contained users.
    \item User data reaches persistence only through the tenant repository.
  \end{itemize}
\end{frame}

\begin{frame}[fragile]{Application Service Coordinates Persistence}
  \begin{lstlisting}
func (s *TenantService) ActivateTenant(ctx context.Context, id TenantID) error {
    tenant, err := s.repo.ByID(ctx, id)
    if err != nil {
        return err
    }
    tenant.Activate()
    return s.repo.Save(ctx, tenant)
}
  \end{lstlisting}
  \begin{itemize}
    \item Application layer owns transaction scope and orchestration.
    \item Repository remains focused on persisting aggregate state.
  \end{itemize}
\end{frame}

\section{DAO vs Repository}

\begin{frame}{Comparing DAO and Repository}
  \begin{tabular}{p{0.3\linewidth}p{0.3\linewidth}p{0.3\linewidth}}
    \textbf{Aspect} & \textbf{DAO} & \textbf{Repository} \\
    \hline
    Focus & Tables / records & Aggregates / roots \\
    Returns & DTOs or raw rows & Domain objects \\
    Layer & Infrastructure & Domain interface, infra implementation \\
    Concern & CRUD mechanics & Aggregate lifecycle \& invariants \\
  \end{tabular}
  \vspace{0.5em}
  \begin{itemize}
    \item Both can coexist: DAO encapsulates data access, repository composes domain-facing contract.
    \item Keeps the domain shielded from persistence technology churn.
  \end{itemize}
\end{frame}

\section{Design Guidance}

\begin{frame}{When to Use a Repository}
  \begin{itemize}
    \item Provide one repository per aggregate root; avoid exposing internal entities.
    \item Shape methods around domain language—\textit{ActivateTenant}, \textit{RegisterUser}, etc.
    \item Let repositories persist lifecycles; avoid mixing analytics or logging concerns.
    \item For cross-aggregate projections, create dedicated read models or query services.
  \end{itemize}
\end{frame}

\begin{frame}{Queries and Consistency}
  \begin{itemize}
    \item Scope repository queries to identifiers or local keys within the aggregate boundary.
    \item Use query services/read models for reporting or multi-aggregate searches.
    \item Embrace CQRS-style separation when projections must remain eventually consistent.
    \item Keeps write-model repositories lean and focused on invariants.
  \end{itemize}
\end{frame}

\section{Testing Strategies}

\begin{frame}[fragile]{Testing Repository Contracts}
  \begin{lstlisting}
type InMemoryTenantRepo struct {
    store map[TenantID]*Tenant
}

func (r *InMemoryTenantRepo) Save(ctx context.Context, t *Tenant) error {
    cp := *t
    r.store[t.ID] = &cp
    return nil
}

func (r *InMemoryTenantRepo) ByID(ctx context.Context, id TenantID) (*Tenant, error) {
    return r.store[id], nil
}
  \end{lstlisting}
  \begin{itemize}
    \item In-memory fakes accelerate unit tests and validate domain logic.
    \item Integration tests with the real database prove mapping fidelity.
    \item Watch for aliasing or shared references in fakes to avoid optimistic tests.
  \end{itemize}
\end{frame}

\section{Key Takeaways}

\begin{frame}{Recap}
  \begin{itemize}
    \item Repositories bridge domain behavior and persistence without leaking infrastructure.
    \item Prefer explicit, persistence-oriented methods in Go; know when collection-style applies.
    \item Keep repository interfaces small, aggregate-focused, and transaction-free.
    \item Separate read models for projections; test with both in-memory and database-backed implementations.
    \item A well-crafted repository sustains the domain between requests—not merely a CRUD wrapper.
  \end{itemize}
\end{frame}

\end{document}
